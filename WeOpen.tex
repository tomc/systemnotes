\documentclass[main]{subfile}
\begin{document}
	
\chapter{Interference Defense (We Open)}
	
	\section[1C]{\cl1}
	
		\subsection[2S and Below]{\sp2 and Below}
		
		Over direct interference below \sp2 we play transfers, every suit bid showing the next suit with 5+ cards and 5+ HCP (good 4 ok). Over spade bids by the opponents, NT becomes a transfer to \ccc. NT bids aren't well defined over non-spade bids by the opponent. 
		
		Notably, transfers into suits the opponent have shown are still natural. We take the general approach the opponents are extremely untrustworthy here, where psyches and misbids are so common. Therefore we just bid our hands without worrying about what they are showing.
		
		Doubles show values with no suit to show. We are allowed to have a 5 card suit that we don't wish to show, but generally this is a more balanced hand.
	
		It is worth noting that all of this applies over double as well. We assume that the auction will likely become more competitive and therefore do not try to relay and what not. 
		
		\subsection{2NT and higher}
		
		When the interference is at 2NT or higher, our bids revert to natural and GF. Double is values, either invitational (6--7) or a GF with no direction. Cuebids are generally 2 suited hands, GF. (Not available below 3NT, so stopper ask doesn't make much sense. Those hands start with double.)
	
	\section[1D]{\di1}
	
		\subsection{General Rules}
		
		Since our \di1 opener says nothing much about the diamond suit, there are some awkward situations in competition where we need to sort out what is going on in the minors.  Here are some good general guidelines:
		
		\begin{itemize}
			\item Most doubles of a \ddd ~cuebid is just showing diamond length.  This is very different from standard contexts where it often is takeout and showing extra strength.  Support doubles do take precedence, but when the level has gotten past a support double than the ``diamond double'' is on.  \di1--(\he1)--Dbl--?  Over \di2 cue the double would be support, but over a \di3 mixed cue, double is diamonds.  With a good \exactshape(31xx) we can pass and double back in.
			\item When NT is not a logical choice as natural it can be used to differentiate hands with just clubs from those with both minors.  For example, \di1--(Dbl)--P--1M; here 1NT would be both minors (5-4 either way) and \cl2 would just be 5 \ccc with any length in \ddd.
		\end{itemize}
	
		\subsection{Low Level Interference}
	
		Over \di1 --(Dbl or \he1) we play a similar transfer based system.
		
		\begin{compbidtable}{1d,x}
			XX & 4+ \hhh, any strength \\
			\he1 & 4--5 \sss \\
			\sp1 & Balanced or both minors. Responder pulls 1NT to show minors. \\
			1NT & Single minor, competitive. \cl2 is pass or correct. \\
			\cl2/\di2 & Natural, forcing 1 round \\
			\he2 & 6+ \sss, any strength \\
			\sp2 & Both minors, mixed strength \\
			2NT & Natural GF, rarely used. \\
		\end{compbidtable}
	
		Bids over \he1 overcall are the same except for XX \& 2NT. There is no redouble to show hearts, but it isn't needed.	2NT is natural and invitational.	
		
		\subsection{Other suit overcalls}
		
		Fairly standard methods. Negative doubles (no upper limit, high level doubles are about strength more than shape)
		
		\subsection{1NT overcall}
				
		``Reverse Capp'':
		
		\begin{compbidtable}{1d,1n}
			Dbl & Penalty \\
			\cl2 & Single suited minor or Minor+Major 2 suiter \\
			\di2 & Both Majors \\
			\he2 & \hhh \\
			\sp2 & \sss \\
		\end{compbidtable}
	
		\begin{compbidtable}{1d,1n,p,p}
			Dbl & 5+ \ddd, 4+ other \\
			New Suit & Nat, denies 5 \ddd.  \cl2 specifically should be 5 clubs. \\
		\end{compbidtable}
	
		\subsection{Misc}
		
		\di1--(Pass)--1M--(1NT), Dbl is still support.
		
		\di1--(Pass)--Pass--(Dbl), pass suggests 4+ \ddd, otherwise bid or XX.
	
		\subsection{Example From Play}	
	
	
	\hrule
		
	\auctionpart{1d,p,1n,2s,?}
	
	Opp interfered RW into our invitational sequence. There are lots of approaches here that are possible, but we decided to play:
	\begin{description}
		\item[Pass] Non-forcing. Expectation is the 1NT bidder will not reopen unless they have a 5 card suit. 
		\item \rem{With Jenni recently I had this auction as the 1NT bidder, I balanced double for takeout. She bid 2NT scrambling. That seems like a fine agreement, I don't see the need to force responder to pass.}
		\item[Dbl] Penalty
		\item[2NT] Mod. Lebensohl: Not a puppet, but instead the 1NT bidder bids their better minor. We may lose out when opener has only clubs, but we will win when they have a 2 suited hand.
		\item[3x] Nat GF
	\end{description}

	\hrule

After we make a support double, new suits are NF. Jump or Q to force.

	
	\section{1M}
	
	\subsection{Takeout Double}
	
	Over 1M-Dbl we play:
	
	\begin{compbidtable}{1M,x}
		\sp1 & Natural (over \he1) \\
		1NT & Xfer to \clubsuit. All xfers promise 5 cards in the suit bid, but can be preparing to raise the major as well. \\
		\cl2 & Xfer to \diamondsuit \\
		\di2 & Xfer to \heartsuit ~(over \sp1) \\
		R-1 & UPH: Limit Raise or better with Min/Max TaJ \\
		& PH: Good Raise of M \\
		2NT & Mixed Raise \\
		JS & Fit \\
		DR & Weak \\
		DJS & Splinter (doesn't promise void) \\
	\end{compbidtable}
	
	\subsection{Other}
	
	See General Rules.
	
	2NT by Responder is natural at their first turn to call.  2NT by opener after raise is ``Good/Bad''; this is true both over our raises and the opponents.
	
	\section{1NT}
	
	We have general defenses for most interference. Where specific agreements exist for specific conventions those take precedence.
	
	General NT defense: Ruebensohl over majors, Lebensohl over minors. Double of \cl2 is Stayman regardless of meaning. \rem{Mar2022} Double of both natural 2 level and 3 level bids are negative/takeout.
	
	Over Opps Dbl, XX forces \cl2 to escape to a minor. Direct bids are ``Systems On''.
	
	Ruebensohl style: 
	
	\begin{compbidtable}{1n,2M}
		\sp2 & NF \\
		2NT & Forces \cl3, auction from here looks like Lebensohl \\
		\cl3 & xfer to \diamondsuit, Inv+. Opener can accept game. \\
		\di3 & xfer to OM, Inv+. \\
		3M & Stayman, no stopper. \\
		3OM & GF \clubsuit \\
		3NT & NF, no stopper \\
	\end{compbidtable}

	\subsection{1NT--\di2 Multi}
	
	We play ``systems on'' over \di2 Multi, with double being a transfer to Hearts. 2NT Puppet becomes the only available Stayman bid.
	
	Pass then double is Penalty. Pass then bid is usually a light hand and/or one not suitable for immediate bidding; by inference, this means direction auctions would imply forward going values; this may not be a firm rule for hands which can get out on the 2-level (major suit xfers), but should always be true of actions which force to the 3-level. For example, an auction like \auctionpart{1n,2d,3c} would not only be a xfer to \ddd, but would also show invitational strength.  
	
	Slower auctions being weak can have unusual meanings; for example, an auction like \auctionpart{1n,2d,p,2h,p,p,2s} is 4 \spadesuit ~and a longer minor. 2NT would be both minors. 
	
	\ifbool{ari}{\rem{T}{There was discussion about playing 2NT here as some form of Lebensohl instead of both minors.  I'm okay with that too, but we need to decide. Until changed the assumption is 3m would be NF natural and a Q would be stopper ask.}}{}
	
	Be cautious about forcing 3NT without stoppers in both majors, often times opener cannot judge when to sit or pull. This includes bids such as \sp2 Size ask. 

	\subsection{1NT--\di2 Majors and similar}
	
	When the opps make a bid showing 2 known suits artificially, we play an Unusual over Unusual style. (Lower for Lower...). Double is general cards and can be the start to a penalty sequence. 2NT isn't Lebensohl, but rather 2 suited for the other suits.
	
	Immediate jumps are stopper showing, denying a stopper in the other suit. To show one of their suits naturally, start with double.
	
	\subsection{1NT--\he2 Majors and similar}
	
	When the opps make a bid showing 2 known suits by bidding one of them (except \cl2), we play double is takeout of the bid suit. The other cheap cuebid is the general cards bid. 2NT starts a Lebensohl sequence. (NOT Ruebensohl)
	
	
	\subsection{1NT--2M M+m and similar}
	
	When the opps show a known and unknown suit by bidding the known we treat the auction as natural. Ruebensohl over Majors, Lebensohl over \ddd, Sys On over \ccc.

	\ifbool{christian}
	{\section[2C]{\cl2}}
	{\section{2m}}
	
	The general style is to treat \ifbool{christian}{this bid}{these bids} much like other openers: negative doubles, new suits forcing after an overcall, etc.
	
	Doubles are ignored; systems on.
	
	After an overcall, doubles are takeout and new suits are forcing 1 round.
	
	\section{Weak 2}
	\ifbool{christian}
	{
	Over double we play ``Transfer McCabe'' with power XX. 2NT is a substitute for clubs/\sp2, hearts/\di2. (That is, the cheapest rank suit that has no easy xfer.)
	
	R-1 is a ``green light'' raise of M, 3M is a ``red light'' raise. Over a ``green light'' raise opener is allowed to further compete if they feel it appropriate.
	}
	{
	Over double, we play ``Transfer McCabe'': XX through R-2 are transfers (with 2NT still being an asking bid) that are either to get out in the next higher suit or a lead directional raise in that suit. R-1 is a ``green light'' raise of M, 3M is a ``red light'' raise. Over a ``green light'' raise opener is allowed to further compete if they feel it appropriate.
	}

	\section{Other}
	
	Transfer McCabe over 3 level preempts as well. No other special agreements.
	
\end{document}