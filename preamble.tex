\documentclass[letterpaper,11pt,oneside]{memoir}
\usepackage{tomc} % reduced version of my bridge.sty
\usepackage{bridgewinners} % taken without modification, but changing things below
\usepackage{fontspec}
\usepackage{unicode-math}  % to allow for new fonts in math mode (i.e., suit symbols)
\usepackage{subfiles} % To have chapters be in separate docs
\usepackage[margin=.2\textwidth]{geometry}
\usepackage{graphicx, tikz}
\usepackage{pgfornament}

\definecolor{clubcolor}{RGB}{23,118,88}
\definecolor{diamondcolor}{RGB}{249,131,10}
\newboolean{bwprint}
\setboolean{bwprint}{true}
\ifthenelse{\boolean{bwprint}}{%
	\standardsuitsymbols}{%
	\colorsuitsymbols}

\setmainfont{DejaVu Serif}
\setmathfont{DejaVu Serif}

\letterten
\chapterstyle{ell}
\renewcommand{\chapterheadstart}{\vspace*{0em}}

\setlength\parindent{0pt} % Ari prefers no indentation for all paragraphs.

\abnormalparskip{1em} % add space between paragraphs for readability

% ToC was not aligning properly, going to modify to get better spacing and readibility based on https://tex.stackexchange.com/questions/35825/pretty-table-of-contents

% a modification of the leftbar environment defined by the framed package
% will be used to place a vertical colored bar separating the page number and the
% title in chapter entries
\renewenvironment{leftbar}{%
	\def\FrameCommand{\textcolor{christian}{\vrule width 2pt depth 6pt}\hspace*{15pt}}%
	\MakeFramed{\advance\hsize-\width\FrameRestore}}%
{\endMakeFramed}

% a command to circle the part numbers
\newcommand\Circle[1]{\tikz[overlay,remember picture] 
	\node[draw=christian,circle, text width=18pt,line width=1pt,align=center] {#1};}

% redefinition of the name of the ToC
\renewcommand\printtoctitle[1]{\HUGE\sffamily\bfseries#1}

\makeatletter
% redefinitions for part entries
\renewcommand\cftpartfont{\Huge\sffamily\bfseries\hfill}
\renewcommand\partnumberline[1]{%
	\hbox to \textwidth{\hss\Circle{\textcolor{christian}{#1}}\hss}%
	\vskip3.5ex\par\hfill\color{tomdark}}
\renewcommand*\cftpartformatpnum[1]{\hfill}
\renewcommand\cftpartafterpnum{\vskip1ex}

% redefinitions for chapter entries
\renewcommand\chapternumberline[1]{\mbox{\small\@chapapp~#1}\par\noindent\Large}
\renewcommand\cftchapterfont{\sffamily}
\cftsetindents{chapter}{0pt}{0em}
\renewcommand\cftchapterpagefont{\Huge\sffamily\bfseries\color{tomdark}}

\newcommand*{\l@mychap}[3]{%
	\def\@chapapp{#3}
	\vskip1ex%
	\par\noindent\begin{minipage}{\textwidth}%
		\parbox{4.5em}{%
			\hfill{\cftchapterpagefont#2}%
		}\hspace*{1.5em}%
		\parbox{\dimexpr\textwidth-4.5em-15pt\relax}{%
			\leftbar\cftchapterfont#1\hspace{1sp}\endleftbar%
		}%
	\end{minipage}\par%
}

\renewcommand*{\l@chapter}[2]{%
	\l@mychap{#1}{#2}{\chaptername}%
}

\renewcommand*{\l@appendix}[2]{%
	\l@mychap{#1}{#2}{\appendixname}%
}

% redefinitions for section entries
\renewcommand\cftsectionfont{\sffamily}
\renewcommand\cftsectionpagefont{\sffamily\itshape\color{tomdark}}
\renewcommand\cftsectionleader{\nobreak}
\renewcommand\cftsectiondotsep{\cftnodots}
\renewcommand\cftsectionafterpnum{\hspace*{\fill}}
\setlength\cftsectionnumwidth{12em}
\cftsetindents{section}{6em}{3em}
\renewcommand\cftsectionformatpnum[1]{%
	\hskip1em\hbox to \@pnumwidth{{\cftsectionpagefont #1\hfill}}}

% redefinitions for subsection entries
\renewcommand\cftsubsectionfont{\sffamily}
\renewcommand\cftsubsectionpagefont{\sffamily\itshape\color{myred}}
\renewcommand\cftsubsectionleader{\nobreak}
\renewcommand\cftsubsectiondotsep{\cftnodots}
\renewcommand\cftsubsectionafterpnum{\hspace*{\fill}}
\setlength\cftsubsectionnumwidth{12em}
\cftsetindents{subsection}{9em}{3em}
\renewcommand\cftsubsectionformatpnum[1]{%
	\hskip1em\hbox to \@pnumwidth{{\cftsubsectionpagefont #1\hfill}}}
\makeatother

%\settocdepth{subsection}
%\setsecnumdepth{subsection}

