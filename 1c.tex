\documentclass[main]{subfile}
\begin{document}
	
	\chapter[1C]{\cl1}
		
	\section{Overview}
	
	\cl1 is our general forcing opening, showing roughly 17+ HCP balanced or 16+ unbalanced.  The new ACBL convention charts allow for as low as 14 HCP for a forcing bid so long as it meets the rule of 24.  (i.e., 10 cards in 2 suits).  I don't know how much we will take advantage of the new rule, but it is worth noting.
	
	Like all strong club systems, the strength of the system lies not within the \cl1 opening itself but rather allowing for all other openers to be limited.  While we have some special sequences which allow for \"{u}ber max in TaJ sequences, most of our simple 1 level openings are going to be lighter than most people open.  All of our invitational and competitive decisions revolve around that fact.  This can be a factor in deciding between \cl1 and 1M, for example.
	
	\begin{bidtable}{1c}
		\di1 & Negative, 0-7(8) \\
		\he1 & 5+\sss, GF\\
		\sp1 & Semi-Bal or 5+\ccc\\
		1NT &  5+\hhh, GF\\
		\cl2 & 5+\ddd, GF\\
		\di2 & 6+\hhh, 3-6\\
		\he2 & 6+\sss, 3-6\\
		\sp2 & (12)13+ Bal (PH: see below)\\
		2NT & (12)13+ (PH: 8-10) \exactshape{1444} \\ 
		\cl3 & (12)13+ (PH: 8-10) \exactshape{4441} (bid sing)\\
		\di3 & (12)13+ (PH: 8-10) \exactshape{4414} (bid sing)\\
		\he3 & (12)13+ (PH: 8-10) \exactshape{4144} (bid sing)\\
		\sp3 & ``Gambling'' hand, AKQxxxx or better. Typically no side cards. \\
		3NT--\he4 & 8+ card transfers, bust hand. No A or K. \\	
	\end{bidtable}

	\section[1C--1D]{\cl1--\di1}
	
	\di1 is the general negative bid.  This is the only non-jump bid which does not set up a GF auction.
	
	Meckwell style rebids except 2NT is 20--21.
	
	\begin{bidtable}{1c,1d}
		\he1 &  4+\hhh, can have a longer minor, 1RF.  Unbalanced or semi-balanced. Systemic rebid with 4=4=(4-1)\\
		\sp1 &  4+\sss, can have a longer minor, 1RF.  Unbalanced or semi-balanced.\\
		1NT & 17--19 bal, can have 5CM or 6Cm.  \shape{5422} also possible.\\
		\cl{2}/\di2 & Nat NF.  Denies 4CM.  Typically 6+ cards and unbalanced.\\
		\he2 & Kokish Relay.  Forces \sp2, Either GF with hearts or GF Bal.\\
		\sp2/\cl3/\di3 & GF Nat, typically 1 suited.\\
		2NT & 20--21 bal\\
		\he3 \& up & Undefined, although game bids are simply to play.		\\
	\end{bidtable}
	\subsection{1M rebid}
	\begin{bidtable}{1c,1d,1h}
		\sp1 & 4+\sss, 3--\hhh, any strength.  Most rebids are natural NF, \\
		& minor suits can be canap\'e. 2NT is an artificial big canap\'e (6+ m) 1RF.\\
		&  Jumps encouraging but NF with jump shifts being 5--5. \\
		1NT & 0--5, no 4CM.  Rebids as per over \sp1, except \sp2 is a natural reverse and 1RF.\\
		\cl2 & 2--\hhh, (5)6 to 7.  \di2 is most minimums (scrambling), \he2 is natural and NF opposite 2=\hhh. Other GF.  2NT is a non-canap\'e GF, 3m is canap\'e.\\
		\di2 & 3=\hhh, (5)6 to 7. \he2 NF, \he3 Inv.  2NT GF asking for shortness NLMH, implies a heart fit. Other 1RF, usually canap\'e.\\
		\he2 & 4+\hhh, minimum. New suits are game tries, 2NT asks shortness NLMH.\\
		2NT & Best raise, nearly GF.  5+\hhh ~common, \cl3 asks for shortness NLMH.\\
		JS & 6+ nat, 5+--7\\
		DJS & Splinter with 4=\hhh \\
	\end{bidtable}
			
	\begin{info}
		Note that over \cl2, 2NT is the non-canap\'e since there can be some natural-ish hands included there, such as 3=5=(4-1).  This is the reverse of the 2NT/JS over \sp1/1NT.
	\end{info}
	
	\orauction{1c,1d,1s} All auctions as per over \he1, except \he2 shows 5+\hhh, (5)6--7.
	\subsection{1NT rebid}
	\orauction{1c,1d,1n} 17--19, systems on as per 1NT opening except secondary xfers such as \\
	\ldots--\di2--\he2--2NT are invitational or better with natural rebids by Opener. (A GF accept of the minor may fake a new suit to keep 3NT in play.)
	
	\subsection{2m rebid}
	
	\orauction{1c,1d,2m} Natural, NF, in principle it denies 4CM.  (Very long minors might suppress the major.) No special follow ups. Jump Shift is a splinter.

	\subsection{JS rebids}
	\cl1--\di1--\he2 is Kokish, forces \sp2.  Either \heartsuit ~or bal, GF.  No agreements about bids other than \sp2 by Responder.
	
	\begin{bidtable}{1c,1d,2h,2s}
		2NT & GF Balanced.  Systems on as per 2NT opener. \\
		\cl3 & \hhh ~\& minor, \di3 for LH. \\
		\di3 & One suited \hhh \\
		\he3 & \hhh ~\& \sss. \sp3 sets spades as trumps, does not show extras. 4m is a Q for hearts, \di4 is last train style. \\
		Other & Sets \hhh~ as trumps, self splinter, demand Q.  (Responder cuebids if able.) \\
	\end{bidtable}

	\ifbool{christian}
	{
		\rem{T}{I think we had a \sp3 rebid in PVD that was intended and interpreted as spades. Memory might be faulty.}
	}{}

	Other jumps are natural GF.  No special agreements other than \ldots\sp2--2NT is a spade raise, with \sp3 being the more waiting nothing-to-say type bid. Typically bal or near bal, 1--2 \sss.
	
	\subsection{2NT rebid}	
	\orauction{1c,1d,2n}  20--21, respond as per 2NT opener.
	
	
	\section[1C--1H]{\cl1--\he1}

	5+\sss, GF.  \sp1 is TaJ with an extra values step by UPH, other bids are natural with no relays.	

	\begin{bidtable}{1c,1h}
		\sp1 & TaJ; 3+\sss ~or 2=\sss ~with extras (20+)\\
		1NT & 17--19 bal or semi-bal with 2--\sss \\
		2x & Natural, no relays \\
		Other & Undefined \\
	\end{bidtable}

		
	\section[1C--1S]{\cl1--\sp1}

	Clubs or balanced, can be semi-balanced.  Notably 4x1 hands without extras start with \sp1 by UPH. (PH shows 4x1 directly, so not included in \sp1 response.)
	
	With \shape{5332} (minor), you can choose between showing your minor (direct \cl2 or \cl2 rebid) or showing a balanced no major (\sp2 rebid). The distinction was more important when \shape{5422} was included in the balanced step, but now that TaJ has been updated to handle that it's probably better to show the \shape{5332} rather than rebid \sp2.
	
	\begin{bidtable}{1c,1s}
		1NT & ``Waiting''; general relay, see below \\
		Other & Naturally, typically 6+ or the higher ranking suit when 5--5. \\
	\end{bidtable}	

	\begin{warning}
		5=\sss~can be tricky if Opener starts with 1NT and Responder bids \di2.  You may have to either bid \sp2 with no slam interest and hope to rebid \sp3 or bid \sp2 directly instead of 1NT, or give up on 5-3 spades and jump to \sp3 to show 4 and COG. 
	\end{warning}

	\begin{bidtable}{1c,1s,1n}
		\cl2 & 5+\ccc, \di2 is TaJ, other natural \\
		\di2 & Balanced 8-11(12) with 4=\hhh \\
		\he2 & Balanced 8-11(12) with 4=\sss ~and 3--\hhh\\
		\sp2 & Balanced 8-11(12) with no 4 card major \\
		2NT & 8-11(12) \exactshape{1444} \\ 
		\cl3 & 8-11(12) \exactshape{4441} \\
		\di3 & 8-11(12) \exactshape{4414} \\		
		\he3 & 8-11(12) \exactshape{4144} \\
	\end{bidtable}

	3 suited hands (2NT--\he3) use the same structure as the direct 3 suiters, where agreeing a suit below game is OKC.

	Over the balanced hands we have options for control relays with or without a fit.
	
	\begin{bidtable}{1c,1s,1n,2d/\hhh/\sss}
		\he2 & (Over \di2) Agrees hearts and asks for controls counting down, 43210 \\
		\sp2 & Shows spades.  If Responder has shown spades then control countdown, 43210. \\
		     & If Responder has bid \di2, then 2NT over \sp2 agrees spades then \cl3 asks controls 43210. \\
		     & \cl3 starts the non-fit 43210 countdown. \\
     2NT & No fits, asks 43210 \\
     \cl3/\ddd & Natural \\
     \he3/\sss & Non-slammish, often choice of games \\
     3NT & NF \\
	\end{bidtable}

	\section[1C--1NT]{\cl1--1NT}
	
	5+\hhh, GF. \cl2 TaJ, \he2 is clubs. Other bids are natural.
	
	\section[1C--2C]{\cl1--\cl2}
	
	5+\ddd, GF. \di2 TaJ. Other bids are natural.
	
	\section{Other}
	
	\subsection{\di2/\hhh ~Semi-Positive Transfers}	
		\di2 and \he2 show 6+ cards in the above major with limited values, about 3--6 HCP.  Not enough to game force but enough length/distribution that game might be in the picture.  Note that 7 HCP not included here since that is generally a GF, but a bad 7 might certainly choose this as an alternative.
		
		Accepting the xfer by Opener is NF, as is 2NT.  Other bids are forcing 1 round.
		
	\subsection{\sp2 Big Balanced (UPH)}
	
	By UPH only, \sp2 shows (12)13+ balanced or 5+ controls.  Only \shape{4333} or \shape{4432} are allowed; with 5 cards suits we show that first then use the extra values step.  With \shape{4441} hands we have direct bids to show that.
		
	2NT is the normal waiting response, with various follows ups.  Anything else is natural and without relay.  It is generally assumed that if Opener does not bid 2NT and Responder bids a new suit that it is a cuebid.
	
	\begin{bidtable}{1c,2s,2n}
		\cl3 & Stayman, with extra-extras, 16+.  Essentially a slam force. \\
		\di3/\hhh& Transfers to 4= majors with 13--15. Accepting the xfer agrees trumps and is OKC. \\
		\sp3 & No major, 13--15 with positive slam interest. \\
		3NT & No major, NF, negative slam interest.  A minimum \sp2 response \\
	\end{bidtable}	

	\rem{T}{Currently undefined for a PH Responder. I recommend we play both \sp2 and 2NT as short spades for memory reasons. \sp2 should be the preferred to not pick off NT, but 2NT should be kept as a memory failsafe.}
	
	\subsection{3 suiters}
	
	2NT thru \he3 are 3 suited hands with shortness in the bid suit. 2NT shows short spades. Over these 4x1 bids, every suit can be agreed below game. 3NT to play.  Agreeing a suit is OKC.
	
	PH changes the range of the bid, but not the nature.
		
	\subsection[3S]{\sp3}
	
	``Gambling'' type hand, AKQ 7th or better with nothing much on the side. Intended to be a picture bid. No special responses at this time. 

	\cl4 asks for shortness, NLMH.  The assumption is that opener knows the suit. 
	
	Currently undefined for a PH Responder
	
	\subsection{8 card busts}
	
	3NT thru \he4 are all 1 under transfers to very long (8+) suits with very weak (no A or K) hands. No special responses.
	
	Note that this bid has \textit{never} actually come up in practice.  \textit{Caveat Lector.}\footnote{Let the Reader Beware}

	\section{PH Changes}
	\begin{itemize}
		\item \sp1 Response does not include 4x1 hands.
		\item \sp2 and \cl3/\ddd/\hhh ~are 4x1 hands. 
		\item 2NT shouldn't be used for \exactshape{1444}, but is preserved for memory reasons.
		\item All TaJ sequences do not include the ``extra values'' step.
	\end{itemize} 

\end{document}




