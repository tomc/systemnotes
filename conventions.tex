\documentclass[conventions]{subfiles}
\begin{document}
	
	\chapter{Conventions by Name}
	
	\section{Kitberry}
	\label{Kitberry}
	
	Kit Woolsey's version of the ``Mulberry'' convention, the key difference being that bids \he4 and higher are natural (and encouraging) and RKC goes through the puppet.  This is an improvement due to being a memory safety net.
	
	Generally is played when there is some 3-suited type bid at the 3-level, to sort out all the possible action follow ups.
	
	The suit order is generally speaking in (known) length order of the 3-suited hand, ties broken up the line.  I'll just refer to the 4 suits as ABCD, where A is the first using that criteria and so on.
	
	\begin{bidtable}{\ldots,\ldots}
		\cl4 & Puppet to \di4 for RKC in ABCD order.  (\he4 = A, etc.) \\
		\di4 & Puppet to \he4 for sign off in any suit \\
		\he4 \& up & Natural, encouraging but NF \\
	\end{bidtable}

	As an example, suppose we are playing this and the auction goes 1NT--\he3.  Since we don't know the order for the minors, they are considered equivalent and broken as clubs before diamonds, with spades next and finally hearts.  So our ABCD would be \ccc\ddd\sss\hhh.
	
	As another possible example of when we might employ such a convention, consider a strong club TaJ auction which shows that Responder is \exactshape{1453}. Since we now all the suits, the ABCD order is just their length: \ddd\hhh\ccc\sss.

\end{document}
