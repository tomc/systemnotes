\documentclass[main]{subfiles}
\begin{document}
	
\chapter[2C]{\cl2}

Our \cl2 opener shows about 10--15 HCP with 6+ \ccc. Good 9 HCPs with some extra shape are acceptable as well, especially in 3rd seat. Four-card or even five-card majors are possible as well. In 3rd seat, opening a good five-card club suit is allowed as well for lead direction/preemptive purposes.

\ifbool{christian}{
	\HUGE\color{red}BETA\normalcolor\normalsize
	
	Since we did play this in PHX I'm marking it as BETA but assuming we are playing it in NOLA.
	
	The general idea is every bid is artificial and generally either the suit bid (natural) with a good hand or a sign off in the next step.
	The reason I wanted to test this structure was similar to why Joel and I played it back in the day, it lets the good hands be played from the strong side while letting you get out with the weaker hands.
	
	For the purposes of below, Opener is allowed to break the "forces" responses with ill fitting hands for the xfer suit. I will not note this in those sections, but it is understood that \textbf{all} of these puppets can be broken. (A few of them it makes no sense to do so that I can see.)
	
	\begin{bidtable}{2c}
		\di2 & \rightarrow\he2, many options \\
		\he2 & \rightarrow\sp2, sign off in \sss ~or inv+ in \hhh \\
		\sp2 & Size ask, inc GF in \sss \\
		2NT & \rightarrow\cl3, weak raise or GF Stayman or 5--5 GF \\
		\cl3 & Constructive Raise \\
		3x & Nearly GF Splinter, \cl4 rebid NF \\
	\end{bidtable}

	% 2d...2h nf, 2s inv, 2NT=jac, 3c=inv, 3d+=d nat rebid, rare: 4h texas
\section[2C--2D]{\cl2--\di2}
	
	\di2 is a multipurpose bid with Responder's rebids showing invitational or more values, depending on the action.
	
	\begin{bidtable}{2c,2d,2h}
		Pass & This is how you escape to \he2 \\
		\sp2 & 5+\sss, inv, NF \\
		2NT & ``Jacoby'', GF \ccc ~raise asking for shortness.  \cl3 is a balanced maximum (12--13), 3NT is a balanced min. (Fast Arrival style) \\
		\cl3 & Light Invite. Sound invites use a direct \sp2 as a Size Ask \\
		\di3+ & Bids above \di3 (except \he4) are natural rebids in the context of a GF \ddd ~response.  \\
		\he4 & Rare, but a NF way to force game in \hhh ~from the Opener's side \\
	\end{bidtable}

\section[2C--2H]{\cl2--\he2}

	One of the simpler responses, this is a pure 2-way bid.  Responder either has \sss ~and is planning on passing \sp2 or they have an invitational or better hand in \hhh. (Or the rare Texas to \sp4.)  
	
	\begin{bidtable}{2c,2h,2s}
		Pass & Escape to \sss \\
		2NT & 5+\hhh, Inv strength, NF but correctable to \cl3 \\
		\sp4 & Texas to \sss \\
		Other & 5+\hhh ~and natural GF bidding \\
	\end{bidtable}

\section[2C--2S]{\cl2--\sp2}

	``Size or Spades''.  Size ask (min/max) but also includes GF Spade hands.
	
	\begin{info}
		Note that \cl2--\he2 has a 3NT rebid to show the \shape{5332}, but here it would be ambiguous as to whether you possess spades or not if you bid 3NT over a size response.  Therefore we use a \di3 bid as an artificial rebid to show 5 only spades.  It is worth noting that there is another sequence which can show 5--5 in \sss ~\& \ddd.  There is no easy way to show 5=\sss ~and 4=\ddd.  I would start with \di3 and let the auction develop from there.
	\end{info}
	
	\begin{bidtable}{2c,2s,2n/3\ccc}
		\cl3 & To play opp min \\
		\di3 & GF with 5=\sss.  This leaves room for Opener to show 4=\hhh, show \sss support or suggest 3NT. \\
		\he3 & 5+\sss, 4=\hhh, GF. We have a different sequence for 5--5 hands. \\
		\sp3 & 6+\sss, GF \\
		3NT & To play \\
		\cl4 & Sets trumps, forcing \\
		\di4 & RKC in \ccc \\		
	\end{bidtable}

\section[2C--2NT]{\cl2--2NT}
	
	2NT is the weaker "preemptive" raise, a GF hand which wants to look for a 4= Major or a 5--5 GF hand.
	\begin{bidtable}{2c,2n,3c}
		Pass & Less than constructive raise \\
		\di3 & Stayman, GF. Natural responses. \\
		\he3 & \hhh ~\& \sss ~GF \\
		\sp3 & \sss ~\& \ddd ~GF \\
		3NT & \ddd ~\& \hhh ~GF \\	
	\end{bidtable}

\section{Other}
	\cl2--\cl3 is a Constructive Raise.  Note that this is \textit{only} constructive, light raises go through \di2. 
	
	\cl2--\di3/\hhh/\sss ~is a splinter raise of clubs. \cl4 rebid by opener is the only non-GF bid.
	
	\cl2--\cl4 is not currently defined, I'd expect undiscussed it would be preemptive but open to other meanings. 
	
	\cl2--\di4 is RKC for \ccc
		
}{}

\ifbool{christian}{\section{Old System}  This is here in case we decide we cannot try the new system out in PHX.}
{ % else
}
\begin{bidtable}{2c}      
\di2 & Artificial asking bid, promising INV+ \\
\he2/\sp2 & NF constructive, usually about 7--11 HCP, 5+ card suit \\
2NT & Puppet to \cl3, showing either a weak raise in \ccc~ (most common) or a GF 5+ 5+ hand without \ccc. \\
\cl3 & Constructive up to a mild invite, usually around 8--11 \\ 
\di3/\he3/\sp3 & Nat, 6+ card suit, GF. \\
3NT & To play. \\
\cl4 & Preemptive \\ 
\di4 & RKC \ccc  \\
\he4/\sp4 & To play. \\
\end{bidtable}

\begin{bidtable}{2c,2d}
\he2 & 4 cards in either major. \sp2 asks, \hhh ~min/\sss ~min/\hhh ~max/\sss ~max. \\
\sp2 & Maximum, no 4-card major, unbalanced or unsuitable for declaring NT. \\
2NT & Maximum, interest in declaring NT. Bal or \shape{6331} with stiff K. \\
\cl3 & Minimum, no 4-card major. \\
\di3 & Maximum, 4+ \ddd \\
Other & Undefined. \\
\end{bidtable}

\cl2--\di2--\sp2--2NT asks for shortness, NLMH.
\end{document}



